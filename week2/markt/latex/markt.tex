\documentclass[a4paper,10pt]{article}

\usepackage[dutch]{babel}
\usepackage{graphicx}
\usepackage{textcomp}
\usepackage{eurosym}

% Packages voor ´Mooie code in LaTeX´
\usepackage{color}
\usepackage{listings}
\usepackage{bold-extra}

% Opening
\title{Markt, Imperatief Programmeren}
\author{Han Kruiger, s1971190 \and Inne Lemstra, s1928473}
\date{\today}

% Definities voor ´Mooie code in LaTeX´
\definecolor{light-gray}{gray}{0.95}
\lstset{
  language=c,				% the language of your code
  basicstyle=\ttfamily,			% the default font
  numbers=left,
  numberstyle=\footnotesize,		% the text - size
  stepnumber=1,				% the step between two line - numbers
  numbersep=5pt,
  backgroundcolor=\color{light-gray},	% background color
  showspaces=false,
  showstringspaces=false,
  showtabs=false,
  frame=single,				% add a frame around the code
  tabsize=2,				% sets default tabsize to 2 spaces
  captionpos=t,
  breaklines=true,
  breakatwhitespace=false,
}

\begin{document}

\maketitle
Docent: Arnold Meijster

\begin{abstract}
Een programma dat de combinaties uitrekent die met drie verschillende vruchten gemaakt kunnen worden, om 100 vruchten te kopen voor een totaalbedrag van \euro100,-.
\end{abstract}

\section{Probleemomschrijving}
Verzogt  wordt een programma te schrijven die de mogelijke combinaties sinaasappels grapefruits en meloenen je kan kopen met 100 euro, waarbij het totaal op 100 vruchten uitkomt.
De gebruiker kan zelf de prijs van de drie vruchten invoeren.
\section{Probleemanalyse}
Het programma dat geschreven moet worden moet alle vruchtencombinaties langsgaan die 100 vruchten opleveren. Daarnaast moet het programma van deze combinaties alleen de combinaties naar het scherm schrijven die zorgen voor een totaalbedrag van \euro 100,-.

Niet alle combinaties die 100 vruchten leveren hoeven echter worden nagegaan. Als een sinaasappel bijvoorbeeld \euro60,- kost, hoef niet gecontroleert te worden of twee sinaasappels en \'e\'en meloen samen precies \euro 100,- kosten; het programma kan stoppen na \'e\'en sinaasappel. 
Eerst wordt nagegaan wat het maximum is van een van de vruchten, om op basis van die gegevens het maximum van de tweede vrucht te bepalen.
Als de eerste beide maxima gevonden zijn is het realatief eenvoudig om het maximum van de derde vrucht te vinden.
Naar het maximum van de derde vrucht hoeft niet lang gezocht te worden, het maximum aantal vruchten is immers bekend, en dus wordt alleen nagegaan of het aantal melonen dat totaal 100 vruchten maakt ook een totaal prijs van \euro 100 oplevert.

Vervolgens wordt het Maximum -1 genomen van de tweede vrucht en  daarmee bepaalt of deze, aangevult met meloen, voldoet aan de 100 vruchten \euro100 regel.
Als alle combinaties van de tweede vrucht gecheckt zijn wordt het aantal van de eerste vruchten met 1 verhoogd en worden ook voor deze alle combinaties bepaalt.
Dit gaat door totdat van het maximum van de eerste vrucht alle combinaties bepaalt zijn.

Op deze manier worden alle combinaties gegeven, als men immers alle combinaties van de eerste twee vruchten heeft vastgesteld moeten dat ook alle combinaties van de derde vrucht zijn.
\section{Ontwerp}
Als eerste worden drie variabelen gedeclareerd, voor de invoer van de stukprijs van de drie vruchten, dit zijn respectievelijk {\tt sinaasappelPrijs grapefruitPrijs en meloenPrijs}.
Verder worden de variabelen maxSinaas, maxGrape en maxMel gedeclareerd voor het tellen van het maximum aantal van de desbetreffende vrucht.
De index voor de sinaasappels en de grapefruits, die als doel heeft de loop te tellen, worden gedeclareerd als sinaasIndex en grapeIndex.
Tenslotte worden er nog variabelen gedeclareerd voor het totaal bedrag uitgegeven per vrucht, dit zijn geldSinaas, geldGrape, geldMel.

vervolgens wordt aan de gebruiker gevraagd de bedragen in eurocenten van de vruchten  in te voeren.
Dit gebeurt uiteraard door een {\tt scanf} functie.

Als alle waarden ingevoerd zijn, wordt gecontroleerd of de prijs van de sinaasappels 0 is.
Mocht dit het geval zijn dan wordt maxSinaas 0, omdat er minimaal een andere vrucht nodig is om op een bedrag van \euro100 te komen.
Is de sinaasappelprijs niet 0 dan is maxSinaas het aantal sinaasappels verkregen met \euro100 \footnote{Het is mogelijk dat dit aantal kan groter is dan 100, hier is rekening mee gehouden in de loop}

Nu volgt een constructie van twee {\tt for-loops}.
In de eerste, die optelt van 0 sinaasappels tot maxSinaas, wordt het bedrag dat besteed wordt aan elke mogelijke aantallen sinaasappels berekent.
Ook wordt gecheckt of de prijs van de grapefruits 0  

\newpage
\section{Programmatekst}
\lstinputlisting[caption=Markt]{markt.c}
\newpage
\section{Testresultaten}
\begin{lstlisting}[caption=Test 1 (voorbeeld uit dictaat)]
Van alle markten thuis
De prijs van een sinaasappel: 35
De prijs van een grapefruit : 60
De prijs van een meloen     : 235

Sinaasappel	Grapefruit	Meloen
1 		76		23 
8 		68		24 
15		60		25 
22		52		26 
29		44		27 
36		36		28 
43		28		29 
50		20		30 
57		12		31 
64		4 		32 
\end{lstlisting}

\begin{lstlisting}[caption=Test 2]
Van alle markten thuis
De prijs van een sinaasappel: 5000
De prijs van een grapefruit : 5
De prijs van een meloen     : 0

Sinaasappel	Grapefruit	Meloen
2 		0 		98 
\end{lstlisting}

\begin{lstlisting}[caption=Test 3]
Van alle markten thuis
De prijs van een sinaasappel: 50
De prijs van een grapefruit : 199
De prijs van een meloen     : 101

Sinaasappel	Grapefruit	Meloen
50		25		25 
\end{lstlisting}
\newpage
\begin{lstlisting}[caption=Test 4]
Van alle markten thuis
De prijs van een sinaasappel: 60
De prijs van een grapefruit : 80
De prijs van een meloen     : 180

Sinaasappel	Grapefruit	Meloen
0 		80		20 
5 		74		21 
10		68		22 
15		62		23 
20		56		24 
25		50		25 
30		44		26 
35		38		27 
40		32		28 
45		26		29 
50		20		30 
55		14		31 
60		8 		32 
65		2 		33 
\end{lstlisting}

\begin{lstlisting}[caption=Test 5]
Van alle markten thuis
De prijs van een sinaasappel: 10
De prijs van een grapefruit : 100
De prijs van een meloen     : 1000

Sinaasappel	Grapefruit	Meloen
0 		100		0 
10		89		1 
20		78		2 
30		67		3 
40		56		4 
50		45		5 
60		34		6 
70		23		7 
80		12		8 
90		1 		9 
\end{lstlisting}
\newpage
\section{Evaluatie}
\section{Evaluatie}
De concreetheid van het probleem, gecombineerd met de geringe informatie die gevraagd wordt van de gebruiker, zorgt voor eenvoudige interpretatie. Het probleem kan slechts op beperkte wijze ge\"interpreteerd worden en is slechts op een enkele wijze eenvoudig uit te voeren, namelijk met behulp van de stelling van Pythagoras.

Van de gebruiker wordt verwacht dat hij de gesloten driehoek bedoelt die wordt gevormd met de drie ingevoerde zijden. Dus geen open driehoek of een driehoek waaruit nog een lijn doorloopt. Het programma is dus niet geschikt voor een geautomatiseerd systeem waarbij dit soort ``driehoeken'' mogelijk voor zouden kunnen komen.

Vanwege de geringe informatie die van de gebruiker gevraagd wordt, enkel de drie zijden van de driehoek, is het gebruiksgemak van het programma hoog. Het kan nog worden verbeterd door een visuele representatie van de driehoek te geven.

In eerdere versies van de code werden drie extra variabelen gebruikt om de invoer te sorteren, maar uiteindelijk bleek \'e\'en extra variabele ({\tt x}) voldoende te zijn. Er is ook gekeken naar een systeem zonder extra variabelen, doch dat zou leiden tot meer statements, wat eindelijk meer processorkracht vergt.

Het is eventueel mogelijk om dit probleem met goniometrische functies op te lossen, dit is echter een omweg en er moeten hiervoor nieuwe wiskundige functies worden ge\"importeerd.

Er is ook nagedacht over of het tot conflicten zou leiden als twee gelijke zijden beide het grootste zouden zijn. Dit bleek niet tot conflicten te leiden, omdat de hoek tegenover beide zijden natuurlijk ook gelijk is.
\end{document}