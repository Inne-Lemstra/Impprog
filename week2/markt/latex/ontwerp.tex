Als eerste worden drie variabelen gedeclareerd, voor de invoer van de stukprijs van de drie vruchten, dit zijn respectievelijk {\tt sinaasappelPrijs grapefruitPrijs en meloenPrijs}.
Verder worden de variabelen maxSinaas, maxGrape en maxMel gedeclareerd voor het tellen van het maximum aantal van de desbetreffende vrucht.
De index voor de sinaasappels en de grapefruits, die als doel heeft de loop te tellen, worden gedeclareerd als sinaasIndex en grapeIndex.
Tenslotte worden er nog variabelen gedeclareerd voor het totaal bedrag uitgegeven per vrucht, dit zijn geldSinaas, geldGrape, geldMel.

vervolgens wordt aan de gebruiker gevraagd de bedragen in eurocenten van de vruchten  in te voeren.
Dit gebeurt uiteraard door een {\tt scanf} functie.

Als alle waarden ingevoerd zijn, wordt gecontroleerd of de prijs van de sinaasappels 0 is.
Mocht dit het geval zijn dan wordt maxSinaas 0, omdat er minimaal een andere vrucht nodig is om op een bedrag van \euro100 te komen.
Is de sinaasappelprijs niet 0 dan is maxSinaas het aantal sinaasappels verkregen met \euro100 \footnote{Het is mogelijk dat dit aantal kan groter is dan 100, hier is rekening mee gehouden in de loop}

Nu volgt een constructie van twee {\tt for-loops}.
In de eerste, die optelt van 0 sinaasappels tot maxSinaas of honderd als dat kleiner is, wordt het bedrag dat besteed wordt aan elke mogelijke aantallen sinaasappels berekent.
Ook wordt gecheckt of de prijs van de grapefruits 0, is dit zo dan is het maximum aantal grapefruits 100 min het de sinaasIndex.
Is dit niet het geval dan is maxGrape het aantal grapefruits dat gekocht kan worden met \euro100 min het geld al besteed aan de sinaasappels.

Vervolgens begint de tweede {\tt for-loop}, die van max