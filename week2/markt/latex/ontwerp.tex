Als eerste worden drie variabelen gedeclareerd, voor de invoer van de stukprijs van de drie vruchten, dit zijn {\tt sinaasappelPrijs}, {\tt grapefruitPrijs} en {\tt meloenPrijs}.
Verder worden de variabelen {\tt maxSinaas}, {\tt maxGrape} en {\tt maxMel} gedeclareerd voor het tellen van het maximale aantal van de desbetreffende vrucht.
De index voor de sinaasappels en de grapefruits, die als doel hebben de loops te tellen, worden gedeclareerd als {\tt sinaasIndex} en {\tt grapeIndex}.
Tenslotte worden er nog variabelen gedeclareerd voor het totaal uitgegeven bedrag per vrucht, dit zijn {\tt geldSinaas}, {\tt geldGrape} en {\tt geldMel}.

Vervolgens wordt aan de gebruiker gevraagd de bedragen van de vruchten in eurocenten in te voeren.
Dit gebeurt uiteraard met de {\tt scanf}-functie.

Als alle waarden ingevoerd zijn, wordt gecontroleerd of de prijs van een enkele sinaasappel 0 is.
Mocht dit het geval zijn dan wordt {\tt maxSinaas} 99, omdat er minimaal een andere vrucht nodig is om op een bedrag van \euro100,- te komen.
Is de sinaasappelprijs niet 0 dan is {\tt maxSinaas} het aantal sinaasappels te verkrijgen met \euro100,-.\footnote{Het is mogelijk dat dit aantal kan groter is dan 100, hier is rekening mee gehouden in de loop.}

Nu volgt een constructie van twee {\tt for}-loops.
In de eerste, die optelt van 0 sinaasappels tot {\tt maxSinaas} (of honderd, als dat kleiner is), wordt het bedrag berekend dat besteed wordt aan elk mogelijk aantal sinaasappels.
Ook wordt gecheckt of de prijs van de grapefruits 0 is, is dit zo dan is het maximum aantal grapefruits 100 $-$ {\tt sinaasIndex}.
Is dit niet het geval, dan is {\tt maxGrape} het aantal grapefruits dat gekocht kan worden met het huidige resterende geld.

Vervolgens begint de tweede {\tt for}-loop, die van 0 tot {\tt maxGrape}, of honderd $-$ {\tt sinaasIndex} loopt, als dat minder is. Hier wordt het bedrag dat besteed wordt aan de mogelijke aantallen grapefruits berekend.
Ook kan het aantal meloenen en het bedrag wat hier aan besteed wordt berekend worden binnen deze lus.
Het maximaal aantal meloenen is het aantal dat je nodig hebt om van het totaal aantal vruchten 100 te maken.

Uiteindelijk word gecontroleerd of het uitgegeven geld, {\tt geldSinaas} + {\tt geld}-\\{\tt Grape} + {\tt geldMel}, precies 100 is.
Of het aantal 100 is hoeft niet gecontroleerd te worden, dit volgt uit de selectie die hierboven gedaan is.
Voor het testen van het aantal meloenen is geen {\tt for}-loop nodig, want {\tt maxMel} is het enige aantal vruchten dat voor een vruchtentotaal van 100 zorgt.
Als er aan de controle wordt voldaan, worden de combinaties geprint.

Vervolgens begint de binnenste loop weer opnieuw en wordt {\tt grapeIndex} 2, nu wordt gecontroleerd of dit voor \euro100,- zorgt. Zo gaat het even door, en worden alle mogelijke combinaties gecontroleerd.


Nu moeten er nog vijf variabelen gedeclareert worden, twee voor de index van de {\tt for-loops} en 3 voor het totaal bedrag van per vrucht.
Er worden maar twee {\tt for-loops} gebruikt in het programma want alsl de eerste twee vruchten vast staan hoeft er maar een combinatie van drie vruchten gecontroleert te worden.
  
Het programma bestaat dus uit 2 {\tt for-loops}, de eerst  bepaalt loopt van 0 tot het maximum aantal sinaasappels (maxSinaas).
Hier wordt het totaal uitgegeven bedrag voor sinaasappels bepaalt, door de stukprijs te vermenigvuldigen met het aantal (sinaasIndex).

