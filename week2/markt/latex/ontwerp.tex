Als eerste worden drie variabelen gedeclareert voer de in voer van de stukprijs van de drie vruchten, dit zijn respectievelijk {\tt sinaasappelPrijs grapefruitPrijs en meloenPrijs}.
vervolgens wordt aan de gebruiker gevraagt de bedragen in eurocenten van de vruchten  in te voeren.
Dit gebeurt uiteraard door een {\tt scanf} functie.

Nu moeten er nog vijf variabelen gedeclareert worden, twee voor de index van de {\tt for-loops} en 3 voor het totaal bedrag van per vrucht.
Er worden maar twee {\tt for-loops} gebruikt in het programma want alsl de eerste twee vruchten vast staan hoeft er maar een combinatie van drie vruchten gecontroleert te worden.
  
