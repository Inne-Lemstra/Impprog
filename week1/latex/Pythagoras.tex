\documentclass[a4paper,10pt]{article}

\usepackage[dutch]{babel}
\usepackage{graphicx}
\usepackage{textcomp}

% Packages voor ´Mooie code in LaTeX´
\usepackage{color}
\usepackage{listings}
\usepackage{bold-extra}

% Opening
\title{Pythagoras, Imperatief Programmeren}
\author{Han Kruiger, s1971190 \and Inne Lemstra, s1928473}
\date{\today}

% Definities voor ´Mooie code in LaTeX´
\definecolor{light-gray}{gray}{0.95}
\lstset{
  language=c,				% the language of your code
  basicstyle=\ttfamily,			% the default font
  numbers=left,
  numberstyle=\footnotesize,		% the text - size
  stepnumber=1,				% the step between two line - numbers
  numbersep=5pt,
  backgroundcolor=\color{light-gray},	% background color
  showspaces=false,
  showstringspaces=false,
  showtabs=false,
  frame=single,				% add a frame around the code
  tabsize=2,				% sets default tabsize to 2 spaces
  captionpos=t,
  breaklines=true,
  breakatwhitespace=false,
}

\begin{document}

\maketitle
Docent: Arnold Meijster

\begin{abstract}
Is een driehoek scherp, rechthoekig, of stomp? Dit programma lost het voor u op.
\end{abstract}

\section{Probleemomschrijving}
Er werd gevraagd naar een programma dat drie lengtes van een driehoek inleest, en van deze waarden analyseert of ze een scherpe, rechthoekige of stompe driehoek vormen.\footnote{Dat wil zeggen, is de hoek tegenover de grootste zijde groter, kleiner, of gelijk aan 90\textdegree?} De lengtes worden door de gebruiker gegeven in gehele centimeters.

\section{Probleemanalyse}

Om de driehoek te analyseren met de stelling van Pythagoras moeten de zijden van de driehoek eerst worden gesorteerd op grootte.

De grootste zijde wordt $c$ genoemd, en de twee resterende zijden $a$ en $b$,wordt de oplossing als volgend gesteld:

\begin{enumerate}
 \item Als $a^2 + b^2 < c^2$ dan is de driehoek stomp.
 \item Als $a^2 + b^2 > c^2$ dan is de driehoek scherp.
 \item Als $a^2 + b^2 = c^2$ dan is de driehoek rechthoekig.
\end{enumerate}

Er wordt vanuit gegaan dat punt 3 (de stelling van Pythagoras) als gegeven wordt beschouwd.

De motivatie achter punt 1 is dat als $c^2$ groter wordt gemaakt dan $a^2 + b^2$, de zijden $a$ en $b$ als het ware uit elkaar getrokken worden. Bij visualisatie is op te merken dat de hoek tussen $a$ en $b$ groter wordt dan 90\textdegree.

Voor punt 2 geldt dezelfde motivatie als voor punt 1, maar dan de andere kant op geredeneerd.

Dit voorgaande gaat er echter vanuit dat er met elke drie lengtes een driehoek gevormd kan worden. Dit hoeft echter niet altijd het geval te zijn! Als zijde $a$ en zijde $b$ samen niet groter zijn dan $c$ kan er geen driehoek gevormd worden.

\section{Ontwerp}
Nadat alle variabelen zijn gedeclareerd moeten de waarden door de gebruiker worden ingevoerd. Dit gebeurd in \'e\'en regel met het commando {\tt scanf(\textquotedblright\%d \%d \%d\textquotedblright, \&a, \&b, \&c)}.

Bij het omzetten van deze probleemanalyse naar code, moeten eerst de zijden van de driehoek aan de juiste variabelen toe worden gewezen: De grootste waarde in de variabele {\tt c}, de andere twee in {\tt a} en {\tt b}. Dit wordt gedaan met een extra vierde variabele, {\tt x}.

Nu alle waarden in de correcte variabelen zijn geplaatst, is de variabele {\tt x} overbodig; deze wordt verder in het programma ook niet meer gebruikt. De andere drie gebruiken worden nog de testen mee uit te voeren.

Voor de test volgorde is het noodzakelijk om eerst te testen of het een geldige driehoek is, dit om onjuistheden te voorkomen. Als een van de {\tt if}-statements {\tt 1} teruggeeft (dus {\tt TRUE}), wordt de bijbehorende uitvoer gegeven met het commando {\tt printf()}.

\newpage

\section{Programmatekst}

% Dit importeert de code uit ´pythagoras.c´.
\lstinputlisting[caption=Pythagoras]{pythagoras.c}

\section{Testresultaten}
\begin{itemize}
 \item Invoer 1 (een rechthoekige driehoek)
\begin{lstlisting}
 3 4 5
\end{lstlisting}
Uitvoer: {\tt Dit is een rechthoekige driehoek.}
\\
\item Invoer 2 (een stomphoekige driehoek)
\begin{lstlisting}
 7 4 5
\end{lstlisting}
Uitvoer: {\tt Dit is een stomphoekige driehoek.}
\\
\item Invoer 3 (een (scherphoekige) gelijkbenige driehoek)
\begin{lstlisting}
 5 3 5
\end{lstlisting}
Uitvoer: {\tt Dit is een scherphoekige driehoek.}
\\
\item Invoer 4 (een gelijkzijdige (en dus scherphoekige) driehoek)
\begin{lstlisting}
 6 6 6
\end{lstlisting}
Uitvoer: {\tt Dit is een scherphoekige driehoek.}
\\
\item Invoer 5 (zijde $a$ en zijde $b$ kunnen elkaar nooit raken)
\begin{lstlisting}
 6 9 2
\end{lstlisting}
Uitvoer: {\tt Dit is geen driehoek.}
\\
\item Invoer 6 (drie punten op \'e\'en lijn)
\begin{lstlisting}
 10 4 6
\end{lstlisting}
Uitvoer: {\tt Dit is geen driehoek.}

\end{itemize}

\newpage

\section{Evaluatie}
De concreetheid van het probleem, gecombineerd met de geringe informatie die gevraagd wordt van de gebruiker, zorgt voor eenvoudige interpretatie. Het probleem kan slechts op beperkte wijze ge\"interpreteerd worden en is slechts op een enkele wijze eenvoudig uit te voeren, namelijk met behulp van de stelling van Pythagoras.

Van de gebruiker wordt verwacht dat hij de gesloten driehoek bedoelt die wordt gevormd met de drie ingevoerde zijden. Dus geen open driehoek of een driehoek waaruit nog een lijn doorloopt. Het programma is dus niet geschikt voor een geautomatiseerd systeem waarbij dit soort ``driehoeken'' mogelijk voor zouden kunnen komen.

Vanwege de geringe informatie die van de gebruiker gevraagd wordt, enkel de drie zijden van de driehoek, is het gebruiksgemak van het programma hoog. Het kan nog worden verbeterd door een visuele representatie van de driehoek te geven.

In eerdere versies van de code werden drie extra variabelen gebruikt om de invoer te sorteren, maar uiteindelijk bleek \'e\'en extra variabele ({\tt x}) voldoende te zijn. Er is ook gekeken naar een systeem zonder extra variabelen, doch dat zou leiden tot meer statements, wat eindelijk meer processorkracht vergt.

Het is eventueel mogelijk om dit probleem met goniometrische functies op te lossen, dit is echter een omweg en er moeten hiervoor nieuwe wiskundige functies worden ge\"importeerd.

Er is ook nagedacht over of het tot conflicten zou leiden als twee gelijke zijden beide het grootste zouden zijn. Dit bleek niet tot conflicten te leiden, omdat de hoek tegenover beide zijden natuurlijk ook gelijk is.

\end{document}