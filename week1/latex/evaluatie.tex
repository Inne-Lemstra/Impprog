\section{Evaluatie}
De concreetheid van het probleem, gecombineerd met de geringe informatie die gevraagd wordt van de gebruiker, zorgt voor eenvoudige interpretatie. Het probleem kan slechts op beperkte wijze ge\"interpreteerd worden en is slechts op een enkele wijze eenvoudig uit te voeren, namelijk met behulp van de stelling van Pythagoras.

Van de gebruiker wordt verwacht dat hij de gesloten driehoek bedoelt die wordt gevormd met de drie ingevoerde zijden. Dus geen open driehoek of een driehoek waaruit nog een lijn doorloopt. Het programma is dus niet geschikt voor een geautomatiseerd systeem waarbij dit soort ``driehoeken'' mogelijk voor zouden kunnen komen.

Vanwege de geringe informatie die van de gebruiker gevraagd wordt, enkel de drie zijden van de driehoek, is het gebruiksgemak van het programma hoog. Het kan nog worden verbeterd door een visuele representatie van de driehoek te geven.

In eerdere versies van de code werden drie extra variabelen gebruikt om de invoer te sorteren, maar uiteindelijk bleek \'e\'en extra variabele ({\tt x}) voldoende te zijn. Er is ook gekeken naar een systeem zonder extra variabelen, doch dat zou leiden tot meer statements, wat eindelijk meer processorkracht vergt.

Het is eventueel mogelijk om dit probleem met goniometrische functies op te lossen, dit is echter een omweg en er moeten hiervoor nieuwe wiskundige functies worden ge\"importeerd.

Er is ook nagedacht over of het tot conflicten zou leiden als twee gelijke zijden beide het grootste zouden zijn. Dit bleek niet tot conflicten te leiden, omdat de hoek tegenover beide zijden natuurlijk ook gelijk is.